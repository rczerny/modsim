%%%%%%%%%%%%%%%%%%%%%%%%%%%%%%%%%%%%%%%%%
% Short Sectioned Assignment
% LaTeX Template
% Version 1.0 (5/5/12)
%
% This template has been downloaded from:
% http://www.LaTeXTemplates.com
%
% Original author:
% Frits Wenneker (http://www.howtotex.com)
%
% License:
% CC BY-NC-SA 3.0 (http://creativecommons.org/licenses/by-nc-sa/3.0/)
%
%%%%%%%%%%%%%%%%%%%%%%%%%%%%%%%%%%%%%%%%%

%----------------------------------------------------------------------------------------
%	PACKAGES AND OTHER DOCUMENT CONFIGURATIONS
%----------------------------------------------------------------------------------------

\documentclass[paper=a4, fontsize=11pt]{scrartcl} % A4 paper and 11pt font size

\usepackage[utf8]{inputenc}
\usepackage[T1]{fontenc} % Use 8-bit encoding that has 256 glyphs
\usepackage{fourier} % Use the Adobe Utopia font for the document - comment this line to return to the LaTeX default
\usepackage[english]{babel} % English language/hyphenation
\usepackage{amsmath,amsfonts,amsthm} % Math packages

\usepackage{lipsum} % Used for inserting dummy 'Lorem ipsum' text into the template

\usepackage{sectsty} % Allows customizing section commands
\allsectionsfont{\centering \normalfont\scshape} % Make all sections centered, the default font and small caps

\usepackage{fancyhdr} % Custom headers and footers
\pagestyle{fancyplain} % Makes all pages in the document conform to the custom headers and footers
\fancyhead{} % No page header - if you want one, create it in the same way as the footers below
\fancyfoot[L]{} % Empty left footer
\fancyfoot[C]{} % Empty center footer
\fancyfoot[R]{\thepage} % Page numbering for right footer
\renewcommand{\headrulewidth}{0pt} % Remove header underlines
\renewcommand{\footrulewidth}{0pt} % Remove footer underlines
\setlength{\headheight}{13.6pt} % Customize the height of the header

\numberwithin{equation}{section} % Number equations within sections (i.e. 1.1, 1.2, 2.1, 2.2 instead of 1, 2, 3, 4)
\numberwithin{figure}{section} % Number figures within sections (i.e. 1.1, 1.2, 2.1, 2.2 instead of 1, 2, 3, 4)
\numberwithin{table}{section} % Number tables within sections (i.e. 1.1, 1.2, 2.1, 2.2 instead of 1, 2, 3, 4)

\setlength\parindent{0pt} % Removes all indentation from paragraphs - comment this line for an assignment with lots of text

%----------------------------------------------------------------------------------------
%	TITLE SECTION
%----------------------------------------------------------------------------------------

\newcommand{\horrule}[1]{\rule{\linewidth}{#1}} % Create horizontal rule command with 1 argument of height

\title{	
\normalfont \normalsize 
\textsc{TU Wien, Modelling and Simulation WS2013} \\ [25pt] % Your university, school and/or department name(s)
\horrule{0.5pt} \\[0.4cm] % Thin top horizontal rule
\huge HPP Lattice-Gas Cellular Automata \\ % The assignment title
\horrule{2pt} \\[0.5cm] % Thick bottom horizontal rule
}

\author{René Czerny - e0825750\\Michael Mayer - e0925636\\Daniel Rubas - e0927260}

\date{\normalsize\today} % Today's date or a custom date

\begin{document}

\maketitle % Print the title

\section{What are Cellular Automata (CA) in general}

%Quelle: http://mathworld.wolfram.com/CellularAutomaton.html
A cellular automaton is a collection of "colored" cells on a grid of specified shape that evolves through a number of discrete time steps according to a set of rules based on the states of neighbouring cells. The rules are then applied iteratively for as many time steps as desired.

%Quelle: 2_3_CA.pdf - chapter 2
CA  can have the following 5 characteristics. Not all of these must be always fulfilled. 

\begin{itemize}
	\item CA are regular arrangements of single cells of the same kind.
	\item Each cell holds a finite number of discrete states.
	\item The states are updated simultaneously (`synchronously') at discrete time levels.
	\item The update rules are deterministic and uniform in space and time.
	\item The rules for the evolution of a cell depend only on a local neighbourhood of cells around it. 
\end{itemize}

%Quelle: http://mathworld.wolfram.com/CellularAutomaton.html
Cellular automata come in a variety of shapes and varieties. One of the most fundamental properties of a cellular automaton is the type of grid on which it is computed. The simplest such "grid" is a one-dimensional line. In two dimensions, square, triangular, and hexagonal grids may be considered. Cellular automata may also be constructed on Cartesian grids in arbitrary numbers of dimensions. 
As our problem needs to be handled with a two-dimensional automata, we will only describe this one here more detailed. 

%Quelle: 2_3_CA.pdf - chapter 2
There is much more freedom for arranging the cells and defining the neighbourhoods for the updating rules in two dimensions.


\section{HPP LG CA}
%Quelle: http://en.wikipedia.org/wiki/Lattice_gas_automaton
lattice gas cellular automata are a type of cellular automaton used to simulate fluid flows.

As a cellular automaton, these models comprise a lattice, where the sites on the lattice can take a certain number of different states. In lattice gas, the various states are particles with certain velocities. Evolution of the simulation is done in discrete time steps. After each time step, the state at a given site can be determined by the state of the site itself and neighboring sites, before the time step.
The state at each site is purely boolean. At a given site, there either is or is not a particle moving in each direction.

In papers published in 1973 and 1976, Hardy, Pomeau and de Pazzis introduced the first Lattice Boltzmann model, which is called the HPP model after the authors. HPP model is a two-dimensional model of fluid particle interactions. In this model, the lattice is square, and the particles travel independently at a unit speed to the discrete time.The particles can move to any of the four sites whose cells share a common edge. Particles cannot move diagonally.
If two particles collide head-on, for example a particle moving to the left meets a particle moving to the right, the outcome will be two particles leaving the site at right angles to the direction they came in.

%Quelle: http://en.wikipedia.org/wiki/HPP_model
In this model the lattice takes the form of a two-dimensional square grid, with particles capable of moving to any of the four adjacent grid points which share a common edge, and particles cannot move diagonally. This means each grid point can only have one of sixteen possible interactions.
\begin{itemize}
	\item Particles exist only on the grid points, never on the edges or surface of the lattice.
	\item Each particle has an associated direction (from one grid point to another immediately adjacent grid point).
	\item Each lattice grid cell can only contain a maximum of one particle for each direction, i.e, contain a total of between zero and four particles.
\end{itemize}
The following rules also govern the model:
\begin{enumerate}
	\item A single particle moves in a fixed direction until it experiences a collision.
	\item Two particles experiencing a head-on collision are deflected perpendicularly.
	\item Two particles experience a collision which isn't head-on simply pass through each other and continue in the same direction.
	\item Optionally, when a particles collides with the edges of a lattice it can rebound.
\end{enumerate}
The HPP models follows a two stage update process.

Collision Step
In this step the above rules, 2., 3. and 4. are checked and applied if any collisions have occurred. This results in head-on collision particles changing direction, pass-through collisions continuing unchanged, or non-colliding particles simple remaining the same.

Transport Step
The second step consists of each particle moving one lattice step in the direction they are currently travelling, which could have been changed by the above Collision Step.

%Quelle: 2_3_CA.pdf - chapter 3
HPP is a two-dimensional lattice-gas cellular automata model over a square
lattice. The vectors ci (i = 1;2;3; 4) connecting nearest neighbours (compare
Fig. 3.1.1) are called lattice vectors or lattice velocities.



\section{implementation rules}
text

\section{screenshots, results and comments}
text

%----------------------------------------------------------------------------------------
%	PROBLEM 1
%----------------------------------------------------------------------------------------

\section{Problem title}

\lipsum[2] % Dummy text

\begin{align} 
\begin{split}
(x+y)^3 	&= (x+y)^2(x+y)\\
&=(x^2+2xy+y^2)(x+y)\\
&=(x^3+2x^2y+xy^2) + (x^2y+2xy^2+y^3)\\
&=x^3+3x^2y+3xy^2+y^3
\end{split}					
\end{align}

Phasellus viverra nulla ut metus varius laoreet. Quisque rutrum. Aenean imperdiet. Etiam ultricies nisi vel augue. Curabitur ullamcorper ultricies

%------------------------------------------------

\subsection{Heading on level 2 (subsection)}

Lorem ipsum dolor sit amet, consectetuer adipiscing elit. 
\begin{align}
A = 
\begin{bmatrix}
A_{11} & A_{21} \\
A_{21} & A_{22}
\end{bmatrix}
\end{align}
Aenean commodo ligula eget dolor. Aenean massa. Cum sociis natoque penatibus et magnis dis parturient montes, nascetur ridiculus mus. Donec quam felis, ultricies nec, pellentesque eu, pretium quis, sem.

%------------------------------------------------

\subsubsection{Heading on level 3 (subsubsection)}

\lipsum[3] % Dummy text

\paragraph{Heading on level 4 (paragraph)}

\lipsum[6] % Dummy text

%----------------------------------------------------------------------------------------
%	PROBLEM 2
%----------------------------------------------------------------------------------------

\section{Lists}

%------------------------------------------------

\subsection{Example of list (3*itemize)}
\begin{itemize}
	\item First item in a list 
		\begin{itemize}
		\item First item in a list 
			\begin{itemize}
			\item First item in a list 
			\item Second item in a list 
			\end{itemize}
		\item Second item in a list 
		\end{itemize}
	\item Second item in a list 
\end{itemize}

%------------------------------------------------

\subsection{Example of list (enumerate)}
\begin{enumerate}
\item First item in a list 
\item Second item in a list 
\item Third item in a list
\end{enumerate}

%----------------------------------------------------------------------------------------

\end{document}